%\begingroup
%\let\cleardoublepage\clearpage


% English abstract
\cleardoublepage
\chapter*{Abstract}
%\markboth{Abstract}{Abstract}
\addcontentsline{toc}{chapter}{Abstract (English/Français/Italiano)} % adds an entry to the table of contents
% put your text here
\vskip0.2cm
%\vspace{0.2cm}
%\vskip0.5cm
%\vspace{0.5cm}
This thesis presents the results of a time-dependent analysis of $\Bz\to\Dmp\pipm$ decays using $3~\rm fb^{-1}$ of 
proton-proton collision data collected
with the LHCb detector at CERN's Large Hadron Collider during Run 1 with a centre-of-mass energy of $7$ (2011) and $8$ (2012) TeV.
The LHCb experiment is dedicated to the study of the properties
of $b$-flavoured hadrons, in particular \CP~violation in the $B$ meson system.
The Standard Model of Particle Physics
describes very precisely the mechanism and the amount of \CP~violation expected in the Universe.
However, the observed matter-antimatter asymmetry is larger by several order of magnitude
compared to the predictions. This could be explained by the existence of a new source of \CP~violation, originating in
New Physics beyond the Standard Model.
The time-dependent analysis of $\Bz\to\Dmp\pipm$ decays provides constraints on
the angle $\gamma$ of the Unitarity Triangle, one of the fundamental parameters
of the Standard Model related to \CP~violation. Since no sizeable high-order Standard Model processes are expected to contribute,
any deviation from the predictions would be an unambiguous signature
of New Physics.
The current experimental precision on $\gamma$ is significantly lower than that of theoretical predictions.
This motivates the effort for new experimental determinations of $\gamma$ in order to reduce its uncertainty.
The analysis of $\Bz\to\Dmp\pipm$ decays, although not as sensitive as that obtained from decays of
charged $B$ mesons into $D^{(*)0}K^{(*)+}$ final states, represents an independent and uncorrelated estimation of $\gamma$
that contributes to the global combination of all $\gamma$ measurements. The result obtained in this thesis is more precise than previous
determinations from other experiments (BaBar, Belle) using $\Bz\to\Dmp\pipm$ decays.
Although based on a very large sample of about half a million signal events, it is still dominated by statistical uncertainties,
indicating good prospects for future improvements in precision with more data from Run 2 and beyond. \\
In addition to the $\Bz\to\Dmp\pipm$ analysis, this thesis also summarizes the studies to improve the
performances of the flavour tagging algorithms used by the LHCb collaboration to infer 
the flavour of neutral $B$ mesons in time-dependent analyses.
The performance of these algorithms, being correlated with the kinematics of the reconstructed particles 
as well as the complexity of the event (number of tracks and primary vertices), showed a significant
decrease on Run 2 data (2015--2018), which were collected at a centre-of-mass energy of $13~\rm TeV$. 
Thanks to new implementations, these algorithms now have a performance similar to that
obtained with Run 1 data. \\   

Keywords: $B$ physics, CKM angle $\gamma$, $CP$ violation, flavour tagging, mixing, LHCb, LHC.

% French abstract
\begin{otherlanguage}{french}
\cleardoublepage
\chapter*{Résumé}
%\markboth{Résumé}{Résumé}
% put your text here
\vskip0.2cm
Cette thèse presente les résultats d'une analyse des désintégrations $\Bz\to\Dmp\pipm$
dans un échantillon de collisions proton-proton de $3~\rm fb^{-1}$ enregistré par le détecteur LHCb 
durant le Run 1 du Grand Collisionneur de Hadrons (LHC)
du CERN, à des énergies dans le centre de masse de 7 (2011) et 8 (2012) TeV.
L'expérience LHCb a pour objet l'étude des propriétés des hadrons lourds, 
et en particulier de la violation de $CP$ (CPV) dans les mésons $B$.
Le Modèle Standard (MS) de la physique des particules décrit très précisément la CPV 
attendue dans l'Univers. Cependant, l'asymétrie observée entre matière et antimatière
dépasse les prédictions de plusieurs ordres de grandeur. Cela pourrait s'expliquer
par l'existence d'une source supplémentaire de CPV au-delà du MS.
L'analyse des désintégrations $\Bz\to\Dmp\pipm$ permet de contraindre l'un
des paramètres fondamentaux de la CPV dans le MS: l'angle $\gamma$ du Triangle d'Unitarité (UT).
Une déviation par rapport aux prédictions
théoriques serait une signe de Nouvelle Physique.
La détermination expérimentale du paramètre $\gamma$ est actuellement biens
moins précise que les prédictions théoriques, ce qui justifie de 
nouvelles mesures de $\gamma$.
Bien que l'analyse des désintégrations $\Bz\to\Dmp\pipm$ soit moins sensible que celle
des désintégrations $\Bp\to D^{(*)0}K^{(*)+}$, elle permet d'estimer $\gamma$
de façon indépendante et ainsi de contribuer à la combinaison globale
de toutes les mesures de ce paramètre.
Le résultat présenté dans cette thèse est plus précis que toutes
les mesures déjà obtenues par d'autres expériences (BaBar et Belle) en
utilisant les désintégrations $\Bz\to\Dmp\pipm$.
Bien que basée sur un demi-million d'événements de signal,
la mesure demeure dominée par les incertitudes statistiques, ce qui ouvre de
belles perspectives pour de futures améliorations de la précision en utilisant
les données du Run 2 et au-delà.

De plus, cette thèse donne un aperçu des 
études réalisées pour améliorer les performances des algorithmes utilisés 
par la collaboration LHCb pour déterminer la saveur des mésons $B$ neutres
étudiés dans les analyses temporelles.
La performance de ces algorithmes, étant tributaire de la cinématique des particules reconstruites
et de la complexité des événements (nombre de traces et de vertex primaires), a
significativement chuté sur les données du Run 2 (2015--2018) 
à une énergie dans le centre de masse de 13 TeV. Gr\^ace à de nouvelles implémentations,
ces algorithmes atteignent de nouveau un niveau de performance similaire à
celui sur les données du Run 1. \\

Mots clefs: physique du meson $B$, angle CKM $\gamma$, violation de $CP$, étiquetage de saveur, oscillations, LHCb, LHC.
%put your text here
\end{otherlanguage}



% Italian abstract
\begin{otherlanguage}{italian}
\cleardoublepage
\chapter*{Sommario}
%\markboth{Zusammenfassung}{Zusammenfassung}
% put your text here
\vskip0.2cm
Questa tesi riporta i risultati di un'analisi dei decadimenti $\Bz\to\Dmp\pipm$ effettuata su $3 \fb^{-1}$ di
dati ottenuti da collisioni protone-protone, i quali sono stati collezionati dal rivelatore LHCb al Grande Collisionatore di Adroni (LHC) durante il Run 1
con un'energia nel centro di massa di 7 (2011) e 8 (2012) TeV. LHCb è dedicato allo studio delle proprietà di adroni
contenenti quark $b$, in particolare la violazione di $CP$ nei mesoni $B$.
Il Modello Standard (MS) della Fisica delle Particelle descrive in modo preciso la 
violazione di $CP$ (CPV) prevista nell'Universo, ma l'asimmetria osservata fra materia ed antimateria è più
grande di diversi ordini di grandezza rispetto alle attese. Ciò può essere spiegato tramite nuovi meccanismi di
CPV non contemplati dal MS. 
L'analisi dei decadimenti $\Bz\to\Dmp\pipm$ consente di ottenere vincoli sull'angolo $\gamma$ del Triangolo di Unitarietà (UT),
un parametro fondamentale del MS legato alla CPV. Poichè nessun contributo apprezzabile è atteso dal MS, ogni
deviazione dalle predizioni di quest'ultimo sarebbe un segnale di Nuova Fisica.
La precisione sperimentale attuale su $\gamma$ è inferiore a quella teorica; ciò motiva gli sforzi per ottenere nuove misure di
$\gamma$ per ridurne l'incertezza. L'analisi dei decadimenti $\Bz\to\Dmp\pipm$, anche se non consente una precisione
paragonabile a quella ottenuta da decadimenti $\Bp\to D^{(*)0}K^{(*)+}$, rappresenta una stima indipendente e
complementare che contribuisce alla combinazione globale di tutte le misure di $\gamma$.
Il risultato ottenuto in questa tesi migliora la precisione di misure precedenti realizzate da altri esperimenti (BaBar, Belle)
nello stesso decadimento.
Nonostante il grande numero di decadimenti analizzati (circa mezzo milione di eventi di segnale), l'incertezza statistica
rappresenta il contributo dominante all'incertezza totale; ciò implica che tale incertezza potrà essere migliorata
utilizzando nuovi dati a partire dal Run 2. \\ 
Oltre all'analisi fin qui descritta, questa tesi riassume gli studi effettuati per migliorare le prestazioni degli
algoritmi di etichettatura utilizzati a LHCb per dedurre il sapore dei mesoni $B$ neutri
nel contesto di analisi del tempo di decadimento. 
Queste prestazioni, correlate alla cinematica delle particelle ed alla complessità
degli eventi (numero di tracce e vertici primari), hanno mostrato un peggioramento significativo
sui dati dal Run 2 (2015--2018), i quali sono stati raccolti con un'energia nel centro di massa pari a 13 TeV.
Grazie a nuove implementazioni di questi algoritmi, le prestazioni sono ora paragonabili a quelle ottenute
sui dati dal Run 1. \\

Parole chiave: fisica del mesone $B$, angolo CKM $\gamma$, violazione di $CP$, etichettatura del sapore, oscillazioni, LHCb, LHC.


%put your text here
\end{otherlanguage}


%\endgroup			
%\vfill
